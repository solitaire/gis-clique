\documentclass[12pt, a4paper]{article}
\usepackage[utf8]{inputenc}
\usepackage{polski} 
\title{\textbf{Poszukiwanie największej kliki w~grafie}}
\author{Anna Stępień, Adam Stelmaszczyk}
\date{}
\begin{document}
\maketitle
\setlength{\parindent}{0in}
\section{Treść zadania}

Kliką danego grafu nazywamy podgraf, w którym każde dwa wierzchołki są ze sobą połączone. 
Należy zaprojektować algorytm do znajdowania największej kliki w danym grafie.

\section{Założenia}

Wejściem dla algorytmu jest graf nieskierowany dany macierzą kwadratową rzędu n:

\vspace{1.5\baselineskip}
$ 
\begin{array}{cccc}
q_{0,0} & q_{1,0} & \ldots & q_{n-1,0} \\
q_{0,1} & q_{1,1} & \ldots & q_{n-1,1} \\
\vdots  & \vdots  & \ddots & \vdots  \\
q_{0,n-1} & q_{1,n-1} & \ldots & q_{n-1,n-1} 
\end{array}
, q_{i,j} \in \{0,1\}, 0 < i,j < n
$
\vspace{1.5\baselineskip}

$q_{i,j}$ równe 0 oznacza, że wierzchołki i oraz j nie są połączone krawędzią. W~przeciwnym razie, wierzchołki są połączone.\\

Wyjściem jest niepusty zbiór numerów wierzchołków, które tworzą największą klikę w podanym grafie.
Wierzchołki numerujemy od 0 do $n-1$.

\section{Referencje}

\end{document}