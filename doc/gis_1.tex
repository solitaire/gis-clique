\documentclass[12pt, a4paper]{article}
\usepackage[utf8]{inputenc}
\usepackage{polski} 
\title{\textbf{Poszukiwanie największej kliki w~grafie}}
\author{Anna Stępień, Adam Stelmaszczyk}
\date{}
\setlength{\parindent}{0in}
\renewcommand\refname{Referencje}

\begin{document}
\maketitle
\section{Treść zadania}

Kliką grafu nazywamy podgraf, w którym każde dwa wierzchołki są ze sobą połączone.
Największą kliką nazywamy klikę o~największej liczbie wierzchołków.
Należy zaprojektować algorytm do znajdowania największej kliki w~danym grafie.

\section{Założenia}

Wejściem dla algorytmu jest graf nieskierowany dany macierzą kwadratową rzędu $n$:

\bigskip
$ 
\begin{array}{cccc}
q_{0,0} & q_{1,0} & \ldots & q_{n-1,0} \\
q_{0,1} & q_{1,1} & \ldots & q_{n-1,1} \\
\vdots  & \vdots  & \ddots & \vdots  \\
q_{0,n-1} & q_{1,n-1} & \ldots & q_{n-1,n-1} 
\end{array}
, q_{i,j} \in \{0,1\}, 0 \leq i,j < n
$
\bigskip

$q_{i,j}$ równe 0 oznacza, że wierzchołki i oraz j nie są połączone krawędzią. W~przeciwnym razie, wierzchołki są połączone.
\par\vspace{\baselineskip}
Macierz jest dana w pliku tekstowym, w którym kolejne $q_{i,j}$ w wierszu $j$ są oddzielone co najmniej jednym znakiem białym. 
Przez znak biały rozumiemy spację lub tabulator. $q_{i,j}$ różne od 0 będą traktowane jak 1.
\par\vspace{\baselineskip}
Wyjściem jest niepusty zbiór numerów wierzchołków, które tworzą największą klikę w~podanym grafie. 
Wierzchołki numerujemy od 0 do $n-1$. W~grafie może istnieć więcej niż jedna największa klika.
W~takim przypadku algorytm zwróci dowolną z~nich.

Realizowana aplikacja będzie pracowała w~trybie konsolowym, z~ewentualną możliwością specyfikacji dodatkowych parametrów.
W~projekcie zostanie wykorzystany algorytm Brona--Kerboscha \cite{bk}.

\bibliographystyle{plain}
\bibliography{references}
\end{document}